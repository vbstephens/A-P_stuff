% Preamble
\documentclass[10pt]{article}

% Packages
\usepackage{writeup_style}
\usepackage{multicol}
\usepackage{hanging}

% Document
\begin{document}
%---------------------------------------------------
% Title, authors and addresses
 \title{\uppercase{Anatomy \& Physiology \\ Word Roots, Prefixes, Suffixes, \& Combining Forms}}
 \date{\vspace{-10ex}}
%\author{Victoria B. Stephens\par
% Created: March 12, 2020 \par
% Last updated: \today}

 \maketitle
% \thispagestyle{empty}
%---------------------------------------------------
%%%%%%%%%%%%%%%%%%%%%%%%%%%%%%%%%%%%%%%%%%%%%%%%%%%%%%%%%%%%%%%%%%%%%%
%


 \begin{multicols}{2}

 \section*{Prefixes}
 \sectionspace

 \footnotesize
 \begin{hangparas}{10pt}{1}
 \textbf{a-, an-} \textit{absence, lack} anaerobic (in the absence of oxygen) \par
 \textbf{ab-} \textit{departing from} abnormal \par
 \textbf{acou-} \textit{hearing} acoustics (the science of sound) \par
 \textbf{ac-, acro-} \textit{extreme, extremity, peak} acrodermatitis (inflammation of the skin of the extremities) \par
 \textbf{ad-} \textit{to, toward} adorbital (toward the orbit) \par
 \textbf{aden-, adeno-} \textit{gland} adeniform (resembling a gland in shape) \par
 \textbf{adren-} \textit{toward the kidney} adrenal gland, located adjacent to the kidney \par
 \textbf{aero-} \textit{air} aerobic respiration \par
 \textbf{af-} \textit{toward} afferent neurons, which carry impulses to CNS \par
 \textbf{agon-} \textit{contest} agonistic and antagonistic muscles \par
 \textbf{alb-} \textit{white} corpus albicans of the ovary (white scar tissue) \par
 \textbf{aliment-} \textit{nourish} alimentary canal (digestive tract) \par
 \textbf{allel-} \textit{of one another} alleles (alternative gene expressions) \par
 \textbf{amphi-} \textit{both, on two sides} amphibian (living both in water and on land)\par
 \textbf{ana-} \textit{apart, up, again} anaphase (phase of mitosis in which chromosomes separate) \par
 \textbf{anastomos-} \textit{come together} arteriovenous anastomosis (connection between an artery and a vein) \par
 \textbf{angi-} \textit{vessel } angiitis (inflammation of a lymph or blood vessel) \par
 \textbf{angin-} \textit{choked } angina pectoris (choked feeling in the chest) \par
 \textbf{ant-, anti-} \textit{opposed to, inhibiting} anticoagulant (substance that prevents blood coagulation) \par
 \textbf{ante-} \textit{preceding, before } antecubital (in front of the elbow) \par
 \textbf{aort-} \textit{great artery } aorta of the heart \par
 \textbf{ap-, api-} \textit{tip, extremity } apex of the heart \par
 \textbf{append-} \textit{hang} appendicular skeleton \par
 \textbf{aqua-, aque-} \textit{water } aqueous solution \par
 \textbf{areola-} \textit{open space } areolar connective tissue \par
 \textbf{arrect-} \textit{upright } arrector pilli muscles, which make hairs stand erect \par
 \textbf{arthr-, arthro-} \textit{joint } arthropathy (any joint disease) \par
 \textbf{artic-} \textit{joint } articular surfaces of bones (connection points) \par
 \textbf{atri-} \textit{vestibule } atria (upper chambers of the heart) \par
 \textbf{auscult-} \textit{listen } ausculatory method for measuring blood pressure \par
 \textbf{aut-, auto-} \textit{self } autogeneous (self-generated) \par
 \textbf{ax-, axi-, axo-} \textit{axis, axle } axial skeleton \par
 \textbf{azyg-} \textit{unpaired } azygous organ (unpaired structure) \par
 \textbf{baro-} \textit{pressure } baroreceptors, which monitor blood pressure \par
 \textbf{bi-} \textit{two } bicuspid (having two cusps) \par
 \textbf{bili-} \textit{bile } bilirubin (a bile pigment) \par
 \textbf{bio-} \textit{life } biology (study of life) \par
 \textbf{blast-} \textit{bud, germ} blastocyte (undifferentiated embryonic cell) \par
 \textbf{brachi-} \textit{arm } brachial plexus of PNS supplies the arm \par
 \textbf{brady-} \textit{slow } bradycardia (abnormally slow heart rate) \par
 \textbf{brev-} \textit{short } fibularis brevis (a short leg muscle) \par
 \textbf{broncho-} \textit{bronchus } bronchospasm (spasmodic contraction of bronchial muscle) \par
 \textbf{bucco-} \textit{cheek } buccolabial (pertaining to cheek and lip) \par
 \textbf{calor-} \textit{heat } calories (unit of energy) \par
 \textbf{capill-} \textit{hair } capillaries (smallest blood vessels) \par
 \textbf{caput-} \textit{head } decapitate (to remove the head)\par
 \textbf{carcin-} \textit{cancer} carcinogenic (causes cancer) \par
 \textbf{cardi-, cardio-} \textit{heart} cardiotoxic (harmful to the heart) \par
 \textbf{carneo-} \textit{flesh } trabeculae carnae (muscle ridges in the ventricles of the heart) \par
 \textbf{carot-} \textit{carrot} carotene (orange pigment) \par
 \textbf{carot-} \textit{stupor} carotid artery, blockage of which causes fainting) \par
 \textbf{cata-} \textit{down } catabolism (chemical breakdown) \par
 \textbf{caud-} \textit{tail } caudal (directional term) \par
 \textbf{cec-} \textit{blind } cecum of large intestine, which is a blind-ended pouch \par
 \textbf{cele-} \textit{abdominal} celiac artery, located in the abdomen \par
 \textbf{cephal-} \textit{head } cephalometer (instrument that measures the head) \par
 \textbf{cerebro-} \textit{brain (esp. cerebrum)} cerebrospinal (of the brain and spinal cord) \par
 \textbf{cervic-} \textit{neck } cervical (of the cervix) \par
 \textbf{chiasm- } \textit{crossing} optic chiasma (the point where optic nerves cross) \par
 \textbf{chole- } \textit{bile } cholecystokinin (a bile-secreting hormone) \par
 \textbf{chondr- } \textit{cartilage} chondrogenic (giving rise to cartilage) \par
 \textbf{chrom- } \textit{colored} chromosomes, which stain darkly \par
 \textbf{cili- } \textit{small hair} ciliated epithelium \par
 \textbf{circum- } \textit{around} circumnuclear (surrounding the nucleus) \par
 \textbf{clavic- } \textit{key } clavicle (a "skeleton key") \par
 \textbf{co-, con- } \textit{together} concentric \par
 \textbf{coccy- } \textit{cuckoo} coccyx, which is beak-shaped \par
 \textbf{coel- } \textit{hollow} coelom (the ventral body cavity) \par
 \textbf{commis- } \textit{united} commissures (connections between the two hemispheres of the brain) \par
 \textbf{contra- } \textit{against, opposite} contraceptive (preventing conception) \par
 \textbf{corn-, cornu- } \textit{horn} stratum corneum (the outer layer of the skin) \par
 \textbf{corp- } \textit{body } corpse \par
 \textbf{cort- } \textit{bark } cortex (the outer layer of the brain and kidney) \par
 \textbf{cost- } \textit{rib } intercostal (between the ribs) \par
 \textbf{crani- } \textit{skull } craniotomy (skull operation) \par
 \textbf{crypt- } \textit{hidden} cryptochidism (non-descent of the testes) \par
 \textbf{cusp- } \textit{pointed} bicuspid valves of the heart \par
 \textbf{cutic- } \textit{skin } cuticle of the nail \par
 \textbf{cyan- } \textit{blue } cyanosis (blue skin color due to lack of oxygen) \par
 \textbf{cyst- } \textit{sac, bladder} cystitis (inflammation of urinary bladder) \par
 \textbf{cyt- } \textit{cell} cytology (study of the cell) \par
 \textbf{de-} \textit{reversal, loss} deactivation \par
 \textbf{decid-} \textit{falling off} deciduous (milk) teeth \par
 \textbf{den-, dent-} \textit{tooth} dentin of the tooth \par
 \textbf{dendr-} \textit{tree, branch} dendrites (branches of the neuron) \par
 \textbf{derm-} \textit{skin} dermis (the deep layer of skin) \par
 \textbf{desm-} \textit{bond} desmosome, which binds adjacent epithelial cells \par
 \textbf{di-} \textit{twice, double} dimorphic (having two forms) \par
 \textbf{dia-} \textit{through, between} diaphragm (wall between two areas) \par
 \textbf{dialys-} \textit{separate, break apart} kidney dialysis, in which waste is removed from the blood \par
 \textbf{diastol-} \textit{stand apart} cardiac diastole (the space between successive contractions of the heart) \par
 \textbf{diure-} \textit{urinate} diuretic (drug that increases urine output) \par
 \textbf{dors-} \textit{the back} dorsal, dorsum \par
 \textbf{duc-, duct-} \textit{lead, draw} ductus deferens (the tube that carries sperm into the urethra) \par
 \textbf{dys-} \textit{difficult, faulty} dyspepsia (disturbed digestion) \par
 \textbf{ec-, ex-, ecto-} \textit{out, outside} excrete (to remove from the body) \par
 \textbf{ectop-} \textit{displaced} ectopic pregnancy, which occurs outside of the uterus \par
 \textbf{edem-} \textit{swelling} edema (accumulation of water in body tissues) \par
 \textbf{ef-} \textit{away} efferent nerve fibers, which carry impulses away from CNS \par
 \textbf{ejac-} \textit{to shoot forth} ejaculation of semen \par
 \textbf{embol-} \textit{wedge} embolus (obstructive object in bloodstream) \par
 \textbf{en-, em-} \textit{in, inside} encysted (enclosed in cyst or capsule) \par
 \textbf{enceph-} \textit{brain} encephalitis (inflammation of the brain) \par
 \textbf{endo-} \textit{within, inner} endocytosis (taking of particles into a cell) \par
 \textbf{entero-} \textit{intestine} enterologist (one who studies intestinal disorders) \par
 \textbf{epi-} \textit{over, above} epidermis (the outer layer of skin) \par
 \textbf{erythr-} \textit{red} erythrocyte (red blood cell) \par
 \textbf{eso-} \textit{within} esophagus \par
 \textbf{eu-} \textit{well} euesthesia (normal state of the senses) \par
 \textbf{excret-} \textit{separate} excretory system \par
 \textbf{exo-} \textit{outside, outer layer} exophthalmos (abnormal protrusion of the eye from the orbit) \par
 \textbf{extra-} \textit{outside, beyond} extracellular (outside the cell) \par
 \textbf{extrins-} \textit{from the outside} extrinsic regulation of the heart \par
 \textbf{fasci-, fascia-} \textit{bundle, band} superficial and deep fascia \par
 \textbf{fenestr-} \textit{window} fenestrated capillaries \par
 \textbf{ferr-} \textit{iron} ferritin (an iron storage protein) \par
 \textbf{flagell-} \textit{whip} flagellum (tail of a sperm cell) \par
 \textbf{flat-} \textit{blow, blown} flatulence \par
 \textbf{folli-} \textit{bag, bellows} hair follicle \par
 \textbf{fontan-} \textit{fountain} fontanelles of the fetal skull \par
 \textbf{foram-} \textit{opening} foramen magnum of the skull \par
 \textbf{foss-} \textit{ditch} fossa ovalis of the heart \par
 \textbf{gam-, gamet-} \textit{married, spouse} gametes (sex cells) \par
 \textbf{gangli-} \textit{swelling, knot} dorsal root ganglia of the spinal nerves \par
 \textbf{gastr-} \textit{stomach} gastrin (a hormone related to gastric acid secretion) \par
 \textbf{germin-} \textit{grow} germinal epithelium of the gonads \par
 \textbf{gero-, geront-} \textit{old man} gerontology (study of aging) \par
 \textbf{gest-} \textit{carried} gestation (the period between conception and birth) \par
 \textbf{glauc-} \textit{gray} glaucoma, which causes gradual blindness \par
 \textbf{glom-} \textit{ball} glomeruli (capillary clusters in the kidneys) \par
 \textbf{glosso-} \textit{tongue} glossopathy (any disease of the tongue) \par
 \textbf{gluco-, glyco-} \textit{sweet} glycolysis (breakdown of glucose) \par
 \textbf{glute-} \textit{buttock} gluteus maximus (largest buttock muscle) \par
 \textbf{gnost-} \textit{knowing} gnostic sense (awareness of self) \par
 \textbf{gompho-} \textit{nail} gomphosis (joint between tooth and jaw) \par
 \textbf{gon-, gono-} \textit{seed, offspring} gonads (the sex organs) \par
 \textbf{gust-} \textit{taste} gustatory sense (sense of taste) \par
 \textbf{hapt-} \textit{fasten, grasp} hapten (a partial antigen) \par
 \textbf{hema-, hemo-} \textit{blood} hematoma (mass of clotted blood) \par
 \textbf{hemato-} \textit{blood} hematocyst (cyst containing blood) \par
 \textbf{hemi-} \textit{half} hemiglossal (pertaining to one half of the tongue) \par
 \textbf{hepta-} \textit{liver} hepatitis (inflammation of the liver) \par
 \textbf{hetero-} \textit{different, other} heterogeneous \par
 \textbf{hiat-} \textit{gap} hiatus of the diaphragm (opening for esophagus) \par
 \textbf{hippo-} \textit{horse} hippocampus of the brain, which is shaped like a seahorse \par
 \textbf{hirsut-} \textit{hairy} hirsutism (excessive body hair) \par
 \textbf{hist-} \textit{tissue} histology (study of tissues) \par
 \textbf{holo-} \textit{whole} holocrine glands, whose secretions are whole cells \par
 \textbf{hom-, homo-} \textit{same} homocentric (having the same center) \par
 \textbf{hormon-} \textit{to excite} hormones \par
 \textbf{humor-} \textit{a fluid} humoral immunity (immunity via antibodies in body fluids) \par
 \textbf{hyal-} \textit{glass, clear} hyaline cartilage, which has no visible fibers \par
 \textbf{hydr-, hydro-} \textit{water} dehydration (loss of body water) \par
 \textbf{hyper-} \textit{excess} hypertension (excessive tension) \par
 \textbf{hypno-} \textit{sleep} hypnosis (sleep-like state) \par
 \textbf{hyster-, hystero-} \textit{uterus, womb} hysterectomy (removal of the uterus) \par
 \textbf{ile-} \textit{intestine} ileum (last portion of the small intestine) \par
 \textbf{im-} \textit{not} impermeable (not permeable) \par
 \textbf{inter-} \textit{between} intercellular (between cells) \par
 \textbf{intercal-} \textit{insert} intercalated discs (connections between heart muscle cells) \par
 \textbf{intra-} \textit{within, inside} intracellular (inside the cell) \par
 \textbf{iso-} \textit{equal, same} isothermal (at the same temperature) \par
 \textbf{jugul-} \textit{throat} jugular veins \par
 \textbf{juxta-} \textit{near, close to} juxtaglomerular complex (cell cluster next to a glomerulus) \par
 \textbf{karyo-} \textit{kernel, nucleus} karyotype (assemblage of nuclear chromosomes) \par
 \textbf{kera-} \textit{horn} keratin (water-repellent protein of the skin) \par
 \textbf{kilo-} \textit{thousand} kilocalorie (one thousand calories) \par
 \textbf{kin-, kines-} \textit{move} kinetic energy \par
 \textbf{labi-, labri} \textit{lip} labial frenulum (membrane joining lip to gum) \par
 \textbf{lact-} \textit{milk} lactose (milk sugar) \par
 \textbf{lacun-} \textit{space, cavity, lake} lacunae (spaces occupied by cartilage and bone cells) \par
 \textbf{lamell-} \textit{small plate} concentric lamellae (bone matrix rings in compact bone) \par
 \textbf{lat-} \textit{wide} latissimus dorsi (a broad back muscle) \par
 \textbf{laten-} \textit{hidden} latent period of a muscle twitch \par
 \textbf{later-} \textit{side} lateral (directional term) \par
 \textbf{leuko-} \textit{white} leukocyte (white blood cell) \par
 \textbf{leva-} \textit{raise, elevate} levator labii superioris (muscles that elevates the upper lip) \par
 \textbf{lingua-} \textit{tongue} lingual tonsil, which is adjacent to the tongue \par
 \textbf{lip-, lipo-} \textit{fat, lipid} lipophage (cell with fat in its cytoplasm) \par
 \textbf{lith-} \textit{stone} cholelithiasis (gallstones) \par
 \textbf{luci-} \textit{clear} stratum lucidum (clear layer of the epidermis) \par
 \textbf{lut-} \textit{yellow} corpus luteum (yellow, hormone-secreting structure in the ovary) \par
 \textbf{macro-} \textit{large} macromolecule (large molecule) \par
 \textbf{magn-} \textit{large} foramen magnum (largest opening of the skull) \par
 \textbf{mal-} \textit{bad, abnormal} malfunction (abnormal functioning of an organ) \par
 \textbf{mamm-} \textit{breast} mammary gland \par
 \textbf{mast-} \textit{breast} mastectomy (removal of a mammary gland) \par
 \textbf{meat-} \textit{passage} external acoustic meatus (the ear canal) \par
 \textbf{medi-} \textit{middle} medial (directional term) \par
 \textbf{medull-} \textit{marrow} medulla (middle portion of the kidney) \par
 \textbf{mega-} \textit{large} megakaryocyte (large precursor cell of platelets) \par
 \textbf{meio-} \textit{less} meiosis (nuclear division that halves the number of chromosomes) \par
 \textbf{melan-} \textit{black} melanocytes, which secrete the black pigment melanin) \par
 \textbf{men-, menstru-} \textit{month} menses (cyclic menstrual flow) \par
 \textbf{meningo-} \textit{membrane} meningitis (inflammation of the membranes of the brain) \par
 \textbf{mer-, mero-} \textit{a part} merocrine glands, whose secretions do not include the cell \par
 \textbf{meso-} \textit{middle} mesoderm (middle germ layer) \par
 \textbf{meta-} \textit{beyond, between} metatarsus (part of the foot between the tarsus and phalanges) \par
 \textbf{metro-} \textit{uterus} endometrium (lining of the uterus) \par
 \textbf{micro-} \textit{small} microscope (instrument for looking at very small things) \par
 \textbf{mictur-} \textit{urinate} micturition (the act of voiding the bladder) \par
 \textbf{mito-} \textit{thread, filament} mitochondria (filament-like structures in cells) \par
 \textbf{mnem-} \textit{memory} amnesia (loss of memory) \par
 \textbf{mono-} \textit{single} monoglyceride (a single-sugar carbohydrate) \par
 \textbf{morpho-} \textit{form} morphology (study of form and structure) \par
 \textbf{muliti-} \textit{many} multinuclear (having several nuclei) \par
 \textbf{mur-} \textit{wall} intramural (within a body or an organ) \par
 \textbf{muta-} \textit{change} mutation (a change in base sequence of DNA) \par
 \textbf{myelo-} \textit{spinal cord, marrow} myeloblasts (bone marrow cells) \par
 \textbf{myo-} \textit{muscle} myocardium (heart muscle) \par
 \textbf{nano-} \textit{dwarf} nanometer (one-billionth of a meter) \par
 \textbf{narco-} \textit{numbness} narcotic (a drug producing stupor or numbed sensations) \par
 \textbf{natri-} \textit{sodium} atrial natrieuretic peptide (a sodium-regulating hormone) \par
 \textbf{necro-} \textit{death} necrosis (tissue death) \par
 \textbf{neo-} \textit{new} neoplasm (an abnormal growth) \par
 \textbf{nephro-} \textit{kidney} nephritis (inflammation of the kidney) \par
 \textbf{neuro-} \textit{nerve} neurophysiology (physiology of the nervous system) \par
 \textbf{noci-} \textit{harmful} nocireceptors (pain receptors) \par
 \textbf{nom-} \textit{name} innominate artery \par
 \textbf{noto-} \textit{back} notochord (embryonic structure preceding the vertebral column) \par
 \textbf{nucle-} \textit{pit, kernel} nucleus \par
 \textbf{nutri-} \textit{feed, nourish} nutrition \par
 \textbf{ob-} \textit{before, against} obstruction \par
 \textbf{oculo-} \textit{eye} monocular (pertaining to one eye) \par
 \textbf{odonto-} \textit{teeth} orthodontist \par
 \textbf{olfact-} \textit{smell} olfactory nerves \par
 \textbf{oligo-} \textit{few} oligodendrocytes (neuroglial cells with few branches) \par
 \textbf{onco-} \textit{a mass} oncology (study of cancer) \par
 \textbf{oo-} \textit{egg} oocyte (precursor of female gamete) \par
 \textbf{ophthalmo-} \textit{eye} ophthalmology (study of the eyes and related diseases) \par
 \textbf{orb-} \textit{circular} orbicularis oculi (muscle that encircles the eye) \par
 \textbf{orchi-} \textit{testis} cryptorchidism (non-descent of the testes) \par
 \textbf{org-} \textit{living} organism \par
 \textbf{ortho-} \textit{straight} orthopedic (correcting of musculoskeletal deformities) \par
 \textbf{osm-} \textit{smell} anosmia (loss of sense of smell) \par
 \textbf{osmo-} \textit{pushing} osmosis \par
 \textbf{osteo-} \textit{bone} osteodermia (bony formations in the skin) \par
 \textbf{oto-} \textit{ear} otoscope (device for examining the ear) \par
 \textbf{ov-, ovi-} \textit{egg} ovum, oviduct \par
 \textbf{oxy-} \textit{oxygen} oxygenation (saturation of substance with oxygen) \par
 \textbf{pan-} \textit{all, universal} panacea (a cure-all) \par
 \textbf{papill-} \textit{nipple} dermal papillae (projections of dermis into epidermis) \par
 \textbf{para-} \textit{beside, near} paranuclear (beside the nucleus) \par
 \textbf{pect-} \textit{breast} pectoralis major (a large chest muscle) \par
 \textbf{pelv-} \textit{a basin} pelvic girdle, which cradles the pelvic organs \par
 \textbf{peni-} \textit{tail} penis \par
 \textbf{penna-} \textit{feather} bipennate muscles, whose fascicles have a feathered appearance \par
 \textbf{pent-} \textit{five} pentose (a 5-carbon sugar) \par
 \textbf{pep-, peps-, pept-} \textit{digest} pepsin (a digestive enzyme) \par
 \textbf{per-, permea-} \textit{through} permeable (can be passed through) \par
 \textbf{peri-} \textit{around} perianal (situated around the anus) \par
 \textbf{phago-} \textit{eat} phagocyte (cell that engulfs and digests particles or cells) \par
 \textbf{pheno-} \textit{show, appear} phenotype (physical appearance) \par
 \textbf{phleb-} \textit{vein} phlebitis (inflammation of the veins) \par
 \textbf{pin-, pino-} \textit{drink} pinocytosis (engulfing of small particles by a cell) \par
 \textbf{platy-} \textit{flat, broad} platysma (a broad, flat neck muscle) \par
 \textbf{pleur-} \textit{side, rib} pleural serosa (membrane lining the thoracic cavity) \par
 \textbf{plex-} \textit{net, network} brachial plexus (network of nerves that supplies the arm) \par
 \textbf{pneumo-} \textit{air} pneumothorax (air in the thoracic cavity) \par
 \textbf{pod-} \textit{foot} podiatry (treatment of foot disorders) \par
 \textbf{poly-} \textit{multiple} polymorphism (having multiple forms) \par
 \textbf{post-} \textit{after, behind} posterior (behind a specific part) \par
 \textbf{pre-, pro-} \textit{before, ahead of} prenatal (before birth) \par
 \textbf{procto-} \textit{rectum, anus} proctoscope (instrument for examining the rectum) \par
 \textbf{pron-} \textit{bent forward} prone, pronate \par
 \textbf{propri-} \textit{one's own} proprioception (awareness of body parts and movement) \par
 \textbf{pseudo-} \textit{false} pseudotumor (false tumor) \par
 \textbf{psycho-} \textit{mind, psyche} psychogram (chart of personality traits) \par
 \textbf{ptos-} \textit{fall} ptosis (upper eyelid droop) \par
 \textbf{pub-} \textit{of the pubis} puberty \par
 \textbf{pulmo-} \textit{lung} pulmonary artery, which carries blood to the lungs \par
 \textbf{pyo-} \textit{pus} pyocyst (a cyst that contains pus) \par
 \textbf{pyro-} \textit{fire} pyrogen (a fever-inducing substance) \par
 \textbf{quad-, quadr-} \textit{four-sided} quadratus lumborum (a square-shaped muscle) \par
 \textbf{re-} \textit{back, again} reinfect \par
 \textbf{rect-} \textit{straight} rectus abdominis, rectum \par
 \textbf{ren-} \textit{kidney} renin (enzyme secreted by the kidney) \par
 \textbf{retin-, retic-} \textit{net, network} endoplasmic reticulum (a network of membranous sacs in a cell) \par
 \textbf{retro-} \textit{backward, behind} retrogression (to move backward in development) \par
 \textbf{rheum-} \textit{watery flow, flux} rheumatoid arthritis, rheumatic fever \par
 \textbf{rhin-, rhino-} \textit{nose} rhinitis (inflammation of the nose) \par
 \textbf{ruga-} \textit{fold, wrinkle} rugae (folds of the stomach, gallbladder, and urinary bladder) \par
 \textbf{sagitt-} \textit{arrow} sagittal (directional term) \par
 \textbf{salta-} \textit{leap} saltatory spasm (sudden, violent, involuntary muscle contraction) \par
 \textbf{sanguin-} \textit{blood} sanguine (the color of dried blood) \par
 \textbf{sarco-} \textit{flesh} sarcomere (unit of contraction in skeletal muscle) \par
 \textbf{saphen-} \textit{visible, clear} great saphenous vein (superficial vein of the thigh and leg) \par
 \textbf{sclero-} \textit{hard} sclerosis (stiffening of a tissue or organ) \par
 \textbf{seb-} \textit{grease} sebum (skin oil) \par
 \textbf{semi-} \textit{half} semicircular \par
 \textbf{sens-} \textit{feeling} sensation, sensory \par
 \textbf{septi-} \textit{rotten} sepsis (infection) \par
 \textbf{sero-} \textit{serum-related} serological tests, which assess blood conditions \par
 \textbf{serrat-} \textit{saw} serratus anterior (a chest wall muscle with a jagged edge) \par
 \textbf{sin-, sino-} \textit{a hollow} sinuses of the skull \par
 \textbf{soma-} \textit{body} somatic nervous system \par
 \textbf{somn-} \textit{sleep} insomnia (inability to sleep) \par
 \textbf{sphin-} \textit{squeeze} sphincter \par
 \textbf{splanchn-} \textit{organ} splanchnic nerve, which supplies the abdominal viscera \par
 \textbf{spondyl-} \textit{vertebra} ankylosing sponylitis (rheumatoid arthritis of the spine) \par
 \textbf{squam-} \textit{scale, flat} squamous epithelium \par
 \textbf{steno-} \textit{narrow} stenosis (abnormal narrowing of a blood vessel or valve) \par
 \textbf{strat-} \textit{layer} strata of the epidermis \par
 \textbf{stria-} \textit{furrow, streak} striations of skeletal and cardiac muscle tissue \par
 \textbf{sub-} \textit{beneath, under} sublingual (beneath the tongue) \par
 \textbf{sucr-} \textit{sweet} sucrose (table sugar) \par
 \textbf{sudor-} \textit{sweat} sudoriferous glands (sweat glands) \par
 \textbf{super-} \textit{above, upon} superior (above something) \par
 \textbf{supra-} \textit{above, upon} supracondylar (above a condyle) \par
 \textbf{sym-, syn-} \textit{together} synapse (region of communication between two neurons) \par
 \textbf{synerg-} \textit{work together} synergism \par
 \textbf{systol-} \textit{contraction} systole (contraction of the heart) \par
 \textbf{tachy-} \textit{rapid} tachycardia (abnormally rapid heartbeat) \par
 \textbf{tact-} \textit{touch} tactile sense \par
 \textbf{telo-} \textit{the end} telophase (last phase of mitosis) \par
 \textbf{templ-, tempo-} \textit{time} temporal summation of nerve impulses \par
 \textbf{tens-} \textit{stretched} muscle tension \par
 \textbf{terti-} \textit{third} fibularis tertius (one of three fibularis muscles) \par
 \textbf{tetan} \textit{rigid} tetanus of the muscles \par
 \textbf{therm-} \textit{heat} thermometer \par
 \textbf{thromb-} \textit{clot} thrombocytopenia, thrombus \par
 \textbf{thyro-} \textit{a shield} thyroid gland \par
 \textbf{tissu-} \textit{woven} tissue \par
 \textbf{tono-} \textit{tension} tonicity, hypertonic \par
 \textbf{tox-} \textit{poison} toxicology (study of poisons) \par
 \textbf{trab-} \textit{beam, timber} trabeculae (spicules of bone in spongy bone tissue) \par
 \textbf{trans-} \textit{across, through} transpleural (through the pleura) \par
 \textbf{trapez-} \textit{table} trapezius (four-sided muscle of the upper back) \par
 \textbf{tri-} \textit{three} trifurcation (division into three branches) \par
 \textbf{trop-} \textit{turn, change} tropic hormones, which target endocrine glands \par
 \textbf{troph-} \textit{nourish} trophoblast, from which the fetal portion of placenta develops \par
 \textbf{tuber-} \textit{swelling} tuberosity (bump on a bone) \par
 \textbf{tunic-} \textit{covering} tunica albuginea (covering of the testis) \par
 \textbf{tympan-} \textit{drum} tympanic membrane (the eardrum) \par
 \textbf{ultra-} \textit{beyond} ultraviolet radiation \par
 \textbf{vacc-} \textit{cow} vaccines, the first of which were made from cowpox \par
 \textbf{vagin-} \textit{a sheath} vagina \par
 \textbf{valen-} \textit{strength} valence shell \par
 \textbf{venter-, ventr-} \textit{abdomen, belly} ventral (directional term), ventricle \par
 \textbf{vent-} \textit{wind} pulmonary ventilation \par
 \textbf{vert-} \textit{turn} vertebral column \par
 \textbf{vestibul-} \textit{a porch} vestibule (anterior entryway to mouth and nose) \par
 \textbf{vibr-} \textit{shake, quiver} vibrissae (hairs of the nasal vestibule) \par
 \textbf{viscero-} \textit{organ, viscera} visceroinhibitory (inhibiting of visceral movement) \par
 \textbf{viscos-} \textit{sticky} viscosity (resistance to flow) \par
 \textbf{vita-} \textit{life} vitamin \par
 \textbf{vitre-} \textit{glass} vitreous humor (clear jelly of the eye) \par
 \textbf{viv-} \textit{live} in vivo \par
 \textbf{vulv-} \textit{a covering} vulva (female external genitalia) \par
 \textbf{zyg-} \textit{a yoke, twin} zygote
 \end{hangparas}

 \normalsize
 \sectionspace
 \section*{Combining Forms}
 \sectionspace
 \footnotesize

 \begin{hangparas}{10pt}{1}
 \textbf{aneurysm} \textit{a widening} aortic aneurysm, in which a weak spot causes enlargement of blood vessel \par
 \textbf{arbor } \textit{tree } arbor vitae of the cerebellum (treelike pattern of white matter) \par
 \textbf{basal } \textit{base } basal lamina of epithelial basement membrane \par
 \textbf{cervix } \textit{neck } cervix of the uterus \par
 \textbf{cochlea } \textit{snail shell} cochlea of the inner ear, which is coiled like a snail shell \par
 \textbf{concha } \textit{shell } nasal conchae (coiled shelves of bone in nasal cavity) \par
 \textbf{corona } \textit{crown } coronal suture of the skull \par
 \textbf{delta } \textit{triangular} deltoid muscle \par
 \textbf{dura } \textit{hard } dura matter (tough outer meninx) \par
 \textbf{gene} \textit{beginning} genetics \par
 \textbf{lamina} \textit{layer, sheet} basal lamina of the epithelial basement membrane \par
 \textbf{lumen} \textit{light} lumen (center of a hollow structure) \par
 \textbf{lymph} \textit{water} lymphatic circulation (return of clear fluid to the bloodstream) \par
 \textbf{macula} \textit{spot} macula lutea (yellow spot on the retina) \par
 \textbf{mater} \textit{mother} dura mater (a membrane that envelops the brain) \par
 \textbf{pectus} \textit{breast} pectoralis major (a large chest muscle) \par
 \textbf{pia} \textit{tender, gentle} pia mater (delicate inner membrane around the brain and spinal cord) \par
 \textbf{pili} \textit{hair} arrector pili muscles of the skin, which make hairs stand erect \par
 \textbf{plexus} \textit{net, network} brachial plexus (network of nerves that supplies the arm) \par
 \textbf{semen} \textit{seed, sperm} semen (discharge of the male reproductive system) \par
 \textbf{septum} \textit{fence} nasal septum \par
 \textbf{stroma} \textit{mattress} stroma (connective tissue framework of some organs) \par
 \textbf{vagus} \textit{wanderer} the vagus nerve, which travels from the brain into the abdominopelvic cavity \par
 \textbf{vas} \textit{vessel, duct} vasoconstriction, vas deferens \par
 \textbf{villus} \textit{shaggy hair} microvilli, which appear like hairs in light microscopy
 \end{hangparas}

 \normalsize
 \sectionspace
 \section*{Suffixes}
 \sectionspace
 \footnotesize

 \begin{hangparas}{10pt}{1}
 \textbf{-able} \textit{able to, capable of} viable (able to exist) \par
 \textbf{-ac} \textit{referring to} cardiac (referring to the heart) \par
 \textbf{-algia} \textit{pain in a certain part} neuralgia (pain along the course of a nerve) \par
 \textbf{-apsi} \textit{juncture} synapse (where two neurons communicate) \par
 \textbf{-ary} \textit{associated with} coronary (associated with the heart) \par
 \textbf{-asthen} \textit{weakness} myastheia gravis (a disease involving paralysis) \par
 \textbf{-blast} \textit{bud, germ} osteoblast \par
 \textbf{-bryo} \textit{swollen} embryo \par
 \textbf{-cide} \textit{destroy or kill} germicide (an agent that kills germs) \par
 \textbf{-cipit} \textit{head} occipital \par
 \textbf{-clast} \textit{break} osteoclast (a cell that dissolves bone matrix) \par
 \textbf{-crine} \textit{separate} endocrine organs, which secrete hormones into the blood \par
 \textbf{-cyte} \textit{cell} osteocyte, adipocyte \par
 \textbf{-dips} \textit{thirst, dry} polydipsia (excessive thirst associated with diabetes) \par
 \textbf{-ectomy} \textit{cutting out} appendectomy (surgical removal of the appendix) \par
 \textbf{-ell, -elle} \textit{small} organelle \par
 \textbf{-emia} \textit{condition of the blood} anemia (deficiency of red blood cells) \par
 \textbf{-esthesi} \textit{sensation} anaesthesia (lack of sensation) \par
 \textbf{-ferrent} \textit{carry} efferent nerves, which carry impulses away from the CNS \par
 \textbf{-form, -forma} \textit{shape} cribriform plate of the ethmoid bone \par
 \textbf{-fuge} \textit{driving out} vermifuge (a substance that expels worms of the intestine) \par
 \textbf{-gen} \textit{an agent that initiates} pathogen (any disease-producing agent) \par
 \textbf{-glea, -glia} \textit{glue} neuroglia (connective tissue of the nervous system) \par
 \textbf{-gram} \textit{a record, data} electrocardiogram (recording showing action of the heart) \par
 \textbf{-graph} \textit{instrument for recording data} electrocardiograph (instrument used to make electrocardiograms) \par
 \textbf{-ia} \textit{condition} insomnia (condition of not being able to sleep) \par
 \textbf{-iatrics} \textit{medical specialty} geriatrics (branch of medicine dealing with old age and associated diseases) \par
 \textbf{-ism} \textit{condition} hyperthyroidism \par
 \textbf{-itis} \textit{inflammation} gastritis (inflammation of the stomach) \par
 \textbf{-lemma} \textit{sheath, husk} sarcolemma (plasma membrane of a muscle cell) \par
 \textbf{-logy} \textit{study of} pathology (the study of disease) \par
 \textbf{-lysis} \textit{loosening, breaking down} hydrolysis (chemical decomposition that takes up water) \par
 \textbf{-malacia} \textit{soft} osteomalacia (a process that leads to bone softening) \par
 \textbf{-mania} \textit{obsession, compulsion} erotomania (exaggeration of the sexual passions) \par
 \textbf{-nata} \textit{birth} prenatal (before birth) \par
 \textbf{-nom} \textit{govern} autonomic nervous system \par
 \textbf{-odyn} \textit{pain} coccygodynia (pain in the coccyx region) \par
 \textbf{-oid} \textit{like, resembling} cuboid (shaped like a cube) \par
 \textbf{-oma} \textit{tumor} lymphoma (a tumor of the lymphatic tissues) \par
 \textbf{-opia} \textit{of the eye} myopia (nearsightedness) \par
 \textbf{-ory} \textit{referring to, of} auditory (related to hearing) \par
 \textbf{-pathy} \textit{disease} osteopathy (any disease of the bone) \par
 \textbf{-phasia} \textit{speech} aphasia (lack of ability to speak) \par
 \textbf{-phil, -philo} \textit{like} hydrophilic (water-loving) \par
 \textbf{-phobia} \textit{fear} acrophobia (fear of heights) \par
 \textbf{-phragm} \textit{partition} diaphragm, which separates the thoracic and abdominal cavities \par
 \textbf{-phylax} \textit{guard, preserve} prophylaxis (to guard in advance, as in preventative treatment) \par
 \textbf{-plas} \textit{grow} neoplasia (an abnormal growth) \par
 \textbf{-plasm} \textit{form, shape} cytoplasm \par
 \textbf{-plasty} \textit{reconstruction of a part} rhinoplasty (surgical reconstruction of the nose) \par
 \textbf{-plegia} \textit{paralysis} paraplegia (paralysis of the lower half of the body or lower limbs) \par
 \textbf{-rrhagia} \textit{flow, discharge} diarrhea (abnormal emptying of the bowels) \par
 \textbf{-scope} \textit{instrument of examination} stethoscope (instrument used to listen to body sounds) \par
 \textbf{-some} \textit{body} chromosome \par
 \textbf{-sorb} \textit{suck in} absorb \par
 \textbf{-stalsis} \textit{compression, constriction} peristalsis (muscular contractions that propel food along the digestive tract) \par
 \textbf{-stasis} \textit{arrest, fixation} hemostasis (arrest of bleeding) \par
 \textbf{-stitia} \textit{come to stand} interstitial fluid, which exists between cells \par
 \textbf{-stomy} \textit{make an artificial opening} enterostomy (formation of an artificial opening into the intestine through the abdominal wall) \par
 \textbf{-tegm} \textit{to cover} integumentary (of the skin and other body coverings) \par
 \textbf{-tomy} \textit{to cut} appendectomy (surgical removal of the appendix) \par
 \textbf{-trud} \textit{thrust} detrusor muscle \par
 \textbf{-ty} \textit{condition of, state} immunity (condition of resistance to infection) \par
 \textbf{-uria} \textit{urine} polyuria (passage of an excessive amount of urine) \par
 \textbf{-zyme} \textit{ferment} enzyme \par
  \end{hangparas}

 \end{multicols}
%%%%%%%%%%%%%%%%%%%%%%%%%%%%%%%%%%%%%%%%%%%%%%%%%%%%%%%%%%%%%%%%%%%%%%


\end{document}